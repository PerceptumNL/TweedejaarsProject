The weighted text vectorizer is an extension of the textvectorizer that takes into account the links of the proposed documents. The vectors of the links of a document are added with some weight to the vectors of the documents themselves. Intuitively, this would add semantic information about a document based on it's links. For example, a Person is likely to write other documents about his or her subjects of expertise. Knowing not only the biography of a Person, but also the content he or she has added to StarFish, gives a more complete image of what documents could be related to that Person.

The vectors of links of a document are added in a recursive way, where documents that are linked directly have a higher weight than documents that are linked transitively. The algorithm is displayed in figure xx.

The weighted text vectorizer performs better than the normal text vectorizer. % LOOK FOR AN EXAMPLE THAT SHOWS THAT THE PROBLEM WAS SOLVED

\subsubsection{Text-based approach limitations}
Overall, both textvectorizers are slow in performance even though the corpus is small. Additionally, the the bag-of-words approach imposes a few limitations on the document linker. Firstly, it performs bad when different languages are used. Figure x shows the vectors of three texts when an English text is used, combined with an English proposed document and a Dutch proposed document. The English and Dutch vectors have only little words in common - luckily the keywords are in this case English, but if they are not this is a problem that cannot be overcome by simply looking at texts. Secondly, the current StarFish network consists of mainly textual content. However, in the future this is likely to be extended with images, videos and other non-textual content. These sources should then somehow be converted to text.