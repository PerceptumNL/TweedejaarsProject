The tag-based approach is more StarFish specific than the text-based approach, since it depends on the tags that are available in StarFish. The tags on StarFish are added by the users themselves, so offer a human-based vision on what a document is really about. 

The simple tag vectorizer is a very straight forward implementation of the idea of using tags. The vectors of this transformation consist of a binary list that tells whether or not a tag is attached to the document. 

Due to it's simplicity, the simple tag vectorizer is very fast. It's performance, as shown in table xx, is about 24\% precision. Both Question documents and Person documents perform quite bad. This can be explained by the fact that half of both Questions and Persons have zero tags. Obviously, the simple tag vectorizer cannot deal with such documents. In fact, almost all other Questions have only one tag. Since the simple tag vectorizer compares vectors, it wil prefer documents that also have only that particular tag, which makes it sensitive to attaching Questions to Questions. Something similar seems to happen with Persons, of which 45\% of the connections are with other Persons. Apparently, persons with similar expertices are tagged similarly. However, as mentioned with the tag vectorizer, in StarFish persons almost never refer to other persons. Moreover, if a document is badly labeled this can also induce problems. For example, take the question 'Is there an English version in Tentamenlade', tagged with 'ToetsenEnToetsgestuurdleren'. The proposed links are visible in table xx, which shows that  the three proposed questions all have the tag 'ToetsEnToetsGestuudLeren'. However, if the question was tagged with the tag 'Tentamenlade', which seems very reasonable given the proposed question, the false negatives would likely be returned correctly by the system. Good practices, events and projects perform significantly better with 75,6\%, 73,0\% and 35,8\%. These are often thoroughly tagged, as shown by document xx, which has a rich set of tags. However, these document types only entail 3.2\%, 2.7\% and 5.4\% of the total amount of documents respectively, which explains why the total performance is still stuck at 24.8\%. 

\begin{lstlisting}
False Positives:
- Wat is het verschil tussen Learning Analytics en TTL (Question)
- Formatieve meerkeuze toetsen om begin kennisniveau te toetsen (Good Practice)
- De toetscyclus (Information)

True Positives:
- Tentamenlade2.5 (Project)

False Negatives:
- Tentamenlade Natuurkunde (Information)
- Hoe kan ik inloggen in Tentamenlade? (Question)
- Hoe kan ik inloggen in Tentamenlade? (Question)
\end{lstlisting}


