
% DO NOT COMPILE THIS FILE DIRECTLY!
% This is included by the other .tex files.

\section{Domain}
The Starfish website aims to be a platform for educators where educational innovations and projects can be shared. Because many teachers have different vocabulary and diverse questions, a strict hierarchical structure of the shared content becomes a problem. Starfish tries to overcome this by structuring it's knowledge base in a non-hierarchical manner. There are subcommunities within Starfish which gives learners the opportunity to share knowledge that is specific to their faculty or institution. 

The knowledge base itself consists of one big set of entities. These entities are called documents. Currently, the Starfish knowledge graph contains 240 documents. These documents can be of a variance of types. Each document in this graph is of one of the following types: Information, Glossary, Question, Good Practice, Project, Person or Event. Documents have an author, title, text, tags and links. The Person type is an exception on that and has a name-field instead of title and a about-field instead of text. Some document types have different optional fields like `headline' for good practices and projects. The graph is structured by directional links between documents.  On average a document x outgoing links, and x incoming links.

Each document in the knowledge base has a set of tags assigned to it, based on the different aspects of educational innovation the document covers. On average a document has x tags. Glossaries are special types of documents, as they are description for tags. This means that it is unneccessary for the link recommendation system to return Glossaries, since a glossary could better be 'linked' via assigning a tag. Because different groups use alternative names to describe concepts, tags can be aliases of each other. For this project, aliased tags are regarded as one single tag. Of the 210 tags the current system contains, x unique tag concepts can be distinguished of which x\% have a glossary. 

These numbers and properties of the system give some insight into the current state of the dataset and possible solutions. The major part of Starfish data is text-based, so semantical document analysis could be peformed with standard text processing techniques. The tags of documents could also give insight into the semantics of the documents. Additionally, the links that are currently in Starfish can be used as guidelines on when documents should be linked and when not. Though Starfish aims to be user-driven, it is unfeasible wiht the current dataset to make the linking user driven. Such an approach could focus on a reinforcement learning system, that learns of the the opinion of users by means of a voting procedure. However, the data needed for this is simply not available. 
