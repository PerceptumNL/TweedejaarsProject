
% DO NOT COMPILE THIS FILE DIRECTLY!
% This is included by the other .tex files.

\section{Domain}
The Starfish website aims to be a platform for educators where educational innovations and projects can be shared. Because many teachers have different vocabulary and diverse questions, a strict hierarchical structure of the shared content is difficult to achieve. Starfish tries to overcome this by structuring it's knowledge base in a non-hierarchical manner. There are sub communities within Starfish which gives learners the opportunity to share knowledge that is specific to their faculty or institution. 

The knowledge base itself consists of one big set of entities. These entities are called documents. Currently, the Starfish knowledge graph contains 240 documents. These documents can be of a variety of types. Each document in this graph is of one of the following types: Information, Glossary, Question, Good Practice, Project, Person or Event. Documents have an `author'-field, `title'-field, `text'-field. The Person-type is an exception, since has a `name'-field instead of `title', an `about'-field instead of `text' and no author. Each document also has a set of tags and links assigned to it. Some document types have different optional fields like `headline' for good practices and projects. The graph is structured by directional links between documents.  On average a document has 3.9 links.

Each document in the knowledge base is assigned a set of tags based on the different aspects of educational innovation the document covers. On average a document has 3.4 tags. Glossaries are special types of documents, as they are description for tags. This means that it is unnecessary for the link recommendation system to return glossaries, since a glossary is `linked' via an assigned tag. Because different groups use alternative names to describe concepts, tags can be aliases of each other. For this project, aliased tags are regarded as one single tag. Of the 210 tags the current system contains, 146 unique tag concepts can be distinguished of which 24.6\% have a glossary. 

These numbers and properties of the system give some insight into the current
state of the dataset and possible solutions. The major part of Starfish data is
text-based, so semantical document analysis could be performed with standard
text processing techniques. The tags of documents could also give insight into
the semantics of the documents. Additionally, the links that are currently in
Starfish can be used as guidelines on when documents should be linked and when
not. 

Although Starfish aims to be user-driven, user voting will not be explored in this 
project. This was not part of the initial request and Starfish is currently to small 
for such an endeavor. The data for this is not at hand and there are too little
users to test such a system effectively. 