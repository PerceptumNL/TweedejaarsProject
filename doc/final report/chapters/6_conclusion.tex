\section{Conclusion}
This report has compared six different ways (vectorizers) of converting a text
document into a vector representation and five ways of computing the distance
between these vectors. All this is done in the context of the Starfish network.
Also a method to take the knowledge from the network into account and a way to
determine the number of documents to retrieve were proposed. These combined
form a complete `pipeline' to compute and propose links to known documents for
a new document that is about to be added to the Starfish network.

The following conclusions can be drawn from the present study: firstly, the
text vectorizers perform significantly better for questions, however these do
perform badly on documents of type `person'. On documents of type `person' the
tag vectorizers do perform well. Suprisingly the simple tag vectorizer has the
best performance and is the fastest except for events. For documents of type
`events' it is recommended to use the glossaries of tags vectorizer.  Secondly
the probabilistic model of the network that is proposed is either to simplistic
or the data available is too little. In either case it might be off interest to
further investigate a similar model on a bigger data set.

% nog iets over threshold

Whilst this study is based on only a small network of document taken together
these findings do show that the chosen path ca


% goed product vision recommender system
